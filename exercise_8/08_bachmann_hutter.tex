%% General definitions
\documentclass{article} %% Determines the general format.
\usepackage{a4wide} %% paper size: A4.
\usepackage[utf8]{inputenc} %% This file is written in UTF-8.
%% Some editors on Windows cannot save files in UTF-8.
%% If there is a problem with special characters not showing up
%% correctly, try switching "utf8" to "latin1" (ISO 8859-1).
\usepackage[T1]{fontenc} %% Format of hte resulting PDF file.
\usepackage{fancyhdr} %% Package to create a header on each page.
\usepackage{lastpage} %% Used for "Page X of Y" in the header.
											%% For this to work, you have to call pdflatex twice.
\usepackage{enumerate} %% Used to change the style of enumerations (see below).

\usepackage{amssymb} %% Definitions for math symbols.
\usepackage{amsmath} %% Definitions for math symbols.
\usepackage{amsthm}
\usepackage{braket}
\usepackage{graphicx}
\usepackage{float}
\usepackage{hyperref}

\usepackage{tikz}  %% Pagacke to create graphics (graphs, automata, etc.)
\usetikzlibrary{automata} %% Tikz library to draw automata
\usetikzlibrary{arrows}   %% Tikz library for nicer arrow heads


%% Left side of header
\lhead{\course\\\semester\\Exercise \homeworkNumber}
%% Right side of header
\rhead{\authorname\\Page \thepage\ of \pageref{LastPage}}
%% Height of header
\usepackage[headheight=36pt]{geometry}
%% Page style that uses the header
\pagestyle{fancy}

\newcommand{\authorname}{Nico Bachmann\\Ruben Hutter}
\newcommand{\semester}{Spring semester 2023}
\newcommand{\course}{Theory of Computer Science}
\newcommand{\homeworkNumber}{8}


\begin{document}

\section*{Exercise \homeworkNumber.1}

KEI PLAN WAS SIE MIT SET $S$ MEINT...

\begin{enumerate}[(a)]
\item Let $M_1$ be a TM that is undefined on input $0$ and $M_2$ be another TM that is defined on input $0$. Let both TM's compute an unary function over $\mathbb{N}$. The Rice theorem tells us that it's impossible to write a general algorithm that decides whether a given TM has a property or not except if the property is always true or false. Since $M_1$ and $M_2$ are two TM's that behave differently, the property of $0$ being undefined is not always true or false and therefore the language L is undecidable.\\

\item Let $M_1$ be a TM that rejects every input and $M_2$ be another TM that accepts every input. Both TM's obviously behave differently and are therefore undecidable using the Rice theorem\\

\item L is decidable because if it halts on an even number of steps for input 0, it always halts on an even number of steps. (KEI PLAN WIESO... CHATGPT)\\

\item L is decidable because we can "simply" compute possible pairs of input and compare them with the output. (EBEFALLS CHATGPT. MACHT IRGENDWIE HALBWEGS SINN??\\
\end{enumerate}

\section*{Exercise \homeworkNumber.2}

\section*{Exercise \homeworkNumber.3}
\begin{enumerate}[(a)]
\item
\item
\end{enumerate}

\section*{Exercise \homeworkNumber.4}
\begin{enumerate}[(a)]
\item
\item
\end{enumerate}


\end{document}
