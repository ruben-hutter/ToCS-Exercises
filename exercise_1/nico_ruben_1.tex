%% General definitions
\documentclass{article} %% Determines the general format.
\usepackage{a4wide} %% paper size: A4.
\usepackage[utf8]{inputenc} %% This file is written in UTF-8.
%% Some editors on Windows cannot save files in UTF-8.
%% If there is a problem with special characters not showing up
%% correctly, try switching "utf8" to "latin1" (ISO 8859-1).
\usepackage[T1]{fontenc} %% Format of hte resulting PDF file.
\usepackage{fancyhdr} %% Package to create a header on each page.
\usepackage{lastpage} %% Used for "Page X of Y" in the header.
											%% For this to work, you have to call pdflatex twice.
\usepackage{enumerate} %% Used to change the style of enumerations (see below).

\usepackage{amssymb} %% Definitions for math symbols.
\usepackage{amsmath} %% Definitions for math symbols.
\usepackage{amsthm}
\usepackage{braket}

\usepackage{tikz}  %% Pagacke to create graphics (graphs, automata, etc.)
\usetikzlibrary{automata} %% Tikz library to draw automata
\usetikzlibrary{arrows}   %% Tikz library for nicer arrow heads


%% Left side of header
\lhead{\course\\\semester\\Exercise \homeworkNumber}
%% Right side of header
\rhead{\authorname\\Page \thepage\ of \pageref{LastPage}}
%% Height of header
\usepackage[headheight=36pt]{geometry}
%% Page style that uses the header
\pagestyle{fancy}

\newcommand{\authorname}{Nico Bachmann\\Ruben Hutter}
\newcommand{\semester}{Spring semester 2023}
\newcommand{\course}{Theory of Computer Science}
\newcommand{\homeworkNumber}{1}


\begin{document}

\section*{Exercise \homeworkNumber.1}
\begin{enumerate}[(a)]
	\item
	At first we write down our different sets of the Domain. Then we combine them
	to form our desired result.\\
	Whe have: $\Sigma_2 \setminus \Sigma_1 = \set{d, e}$, and $Q^2 =
	\set{(q_1, q_1), (q_1, q_2), (q_2, q_1), (q_2, q_2)}$.\\
	If we make the scalar product of this two sets we obtain the following result:\\
	$$R = \set{((q_1, q_1), d), ((q_1, q_2), e), ((q_2, q_1), d), ((q_2, q_2), e),
	((q_1, q_1), e)}$$\\

	\item
	First of all we write down the domain, that is also the power set of
	$\Sigma_1 \cap \Sigma_2$:
	$$\mathcal{P}(\Sigma_1 \cap \Sigma_2) = \set{\emptyset, \set{b}, \set{c},
	\set{b, c}}$$ and the codomain:
	$$Q \times V = \set{(q_1, X), (q_1, Y), (q_1, Z), (q_2, X), (q_2, Y), (q_2, Z)}$$
	An example for a total function could be:
	$$f: \set{\emptyset \mapsto (q_1, X), \set{b} \mapsto (q_2, X), \set{c} \mapsto (q_2, Z), \set{b, c} \mapsto (q_1, Y)}$$\\

	\item
	We want to calculate how many partial functions $f: \Sigma_1 \cup \Sigma_2 \to_p V$
	there are.\\
	The domain of the function $f$ is: $$\Sigma_1 \cup \Sigma_2 =
	\set{a, b, c, d, e}$$
	and the codomain: $$V = \set{X, Y, Z}$$
	Since we have 4 possible mappings for every element of the set (the 3 variables and the emptyset) $\Sigma_1 \cup \Sigma_2$, with cardinality $|\Sigma_1 \cup \Sigma_2| = 5$,
	we have $4^5$ possible partial functions.
\end{enumerate}

\clearpage

\section*{Exercise \homeworkNumber.2}
\begin{enumerate}[(a)]
	\item
	Just for clarity we write the set of all cells: $P = \set{(1,1), (1,2),
	(2,1), \dots, (n,m)}$\\
	In the given example we have: $n = 3$, $m = 2$, $C = \set{\text{White, Gray,
	Black}}$ and the following function $f$:
	$$
	f(P_{3 \times 2}) :=
	\begin{cases}
		Grey & P = (1, 1)\\
		Black & P = (3, 2)\\
		White & else
	\end{cases}
	$$\\

	\item
	The cartesian product of $P \times P$ results in:
	$$P^2 = \set{((1,1), (1,1)), ((1,1), (1,2)), \dots, ((n,m), (n,m))}$$
	and therefore the relation $R$:
	$$R = \set{(p, p') = ((x_p, y_p), (x_{p'}, y_{p'})) \mid f(p) \neq f(p'),\;
	x_p = x_{p'} \text{ and } y_p = y_{p'} + 1}$$

	R for concrete example in (a):
	$$R = \set{((1,2), (1,1)), ((3,2), (3,1))}$$
\end{enumerate}

\clearpage

\section*{Exercise \homeworkNumber.3}
The statement to prove is:
$$(A \cup B) \subseteq (A \cap B) \text{, then } A \subseteq B$$
\begin{enumerate}[(a)]
	\item
	\begin{proof}[Direct proof]
		In a first step we take an $x \in (A \cup B)$. Because $A \cup B$ is a
		subset of $A \cap B$, any $x \in A$ is always also element of $B$ and
		vice versa, and therefore $A \subseteq B$.\\
	\end{proof}

	\item
	\begin{proof}[Indirect proof]
		Assumption: if $(A \cup B) \subseteq (A \cap B)$, then $A \nsubseteq B$\\
		There must be an $x \in A \setminus B$, that is also element of $A \cup B$.
		Because $(A \cup B) \subseteq (A \cap B)$ and $x \notin (A \cap B)$,
		following the initial condition of our $x$, $x$ cannot be in $A$ which
		is a contraddiction to our assumption.\\
	\end{proof}

	\item
	\begin{proof}[Contrapositive]
		We assume: if $A \nsubseteq B$, then $(A \cup B) \nsubseteq (A \cap B)$\\
		Because of $A \nsubseteq B$, there must be an $x \in A \setminus B$ which
		contradicts with $(A \cup B) \subseteq (A \cap B)$ because if $x \in A
		\setminus B$, so also $x \in A \cup B$ but $x$ is not element of $A \cap B$.\\
	\end{proof}
\end{enumerate}

\clearpage

\section*{Exercise \homeworkNumber.4}
\begin{proof}
	For $a \in \mathbb{R}$ and $a \neq 1$\\\\
	\underline{Basis n = 0}:
	$$
	\sum_{i = 0}^{0} a^0 = a^0 = 1 \stackrel{!}{=} \frac{1 - a}{1 - a} = 1
	$$
	\underline{IH}:
	$$
	\sum_{i = 0}^{n} a^i = \frac{1 - a^{n + 1}}{1 - a} \qquad \forall n \in \mathbb{N}_0
	$$
	\underline{Inductive step n $\to$ n + 1}:
	\begin{align*}
		\sum_{i = 0}^{n + 1} a^i = \sum_{i = 0}^{n} a^i + a^{n + 1} &\stackrel{IH}{=} \frac{1 - a^{n + 1}}{1 - a} + a^{n + 1} \\
			&= \frac{1 - a^{n + 1} + (1 - a)a^{n + 1}}{1 - a} \\
			&= \frac{1 - a^{n + 1} + a^{n + 1} - a^{n + 2}}{1 - a} \\
			&= \frac{1 - a^{n + 2}}{1 - a}
	\end{align*}
\end{proof}
\end{document}
