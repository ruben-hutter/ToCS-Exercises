%% General definitions
\documentclass{article} %% Determines the general format.
\usepackage{a4wide} %% paper size: A4.
\usepackage[utf8]{inputenc} %% This file is written in UTF-8.
%% Some editors on Windows cannot save files in UTF-8.
%% If there is a problem with special characters not showing up
%% correctly, try switching "utf8" to "latin1" (ISO 8859-1).
\usepackage[T1]{fontenc} %% Format of hte resulting PDF file.
\usepackage{fancyhdr} %% Package to create a header on each page.
\usepackage{lastpage} %% Used for "Page X of Y" in the header.
											%% For this to work, you have to call pdflatex twice.
\usepackage{enumerate} %% Used to change the style of enumerations (see below).

\usepackage{amssymb} %% Definitions for math symbols.
\usepackage{amsmath} %% Definitions for math symbols.
\usepackage{amsthm}
\usepackage{braket}
\usepackage{graphicx}
\usepackage{float}

\usepackage{tikz}  %% Pagacke to create graphics (graphs, automata, etc.)
\usetikzlibrary{automata} %% Tikz library to draw automata
\usetikzlibrary{arrows}   %% Tikz library for nicer arrow heads


%% Left side of header
\lhead{\course\\\semester\\Exercise \homeworkNumber}
%% Right side of header
\rhead{\authorname\\Page \thepage\ of \pageref{LastPage}}
%% Height of header
\usepackage[headheight=36pt]{geometry}
%% Page style that uses the header
\pagestyle{fancy}

\newcommand{\authorname}{Nico Bachmann\\Ruben Hutter}
\newcommand{\semester}{Spring semester 2023}
\newcommand{\course}{Theory of Computer Science}
\newcommand{\homeworkNumber}{5}


\begin{document}

\section*{Exercise \homeworkNumber.1}
\textbf{Step 1: $\epsilon$ Rules}\\
$S \to \epsilon$ is the only occurrence of $\epsilon$ and $S$ never occurs on the right-hand side. We don't have to do anything here.\\\\
\textbf{Step 2: Chain Rules}\\
Since the CMS (WHAT DO YOU MEAN BY THAT??? $\to$ CNF?) can only have a terminal or \textbf{two} Variables on the right-hand side, we have to eliminate every Rule that has only a single Variable on the right.\\\\
Chains in $G$:
$$S \to Z \to bb \qquad \qquad S \to Z \to Za \qquad \qquad X \to Y \to bY \qquad \qquad Y \to X \to aZb$$
We eliminate the chains by linking the end result of the chain directly with the first Variable. Since $X \to Y$ and $Y \to X$, we can just write them together as $X$ that points to the results of $X$ and $Y$.
$$S \to bb \mid Za \mid XX \qquad \qquad Z \to bb \mid Za \qquad \qquad X \to aZb \mid bX$$
\textbf{Step 3: Order}\\
Now we have to eliminate all the rules that have terminals \textbf{and} Variables on the right-hand side. We do so by adding a new Variable for each terminal and having each new Variable point to exactly one terminal.\\
So we create the new rules:
$$A \to a \qquad \text{ and } \qquad B \to b$$
and can now rewrite the Rules above to:
$$S \to BB \mid ZA \mid XX \qquad \qquad Z \to BB \mid ZA \qquad \qquad X \to AZB \mid BX$$
\textbf{Step 4: Shorten}\\
To get our CNF we have to shorten the right-hand side to exactly two variables per rule. Since $X \to AZB$ is too long, we add $V \to ZB$ and get our final Grammar $G' = \left \langle \set{S, V, X, Z}, \set{a,b}, P, S \right \rangle$ with the following rules in $P$:
$$S \to BB \mid ZA \mid XX \qquad \qquad Z \to BB \mid ZA \qquad \qquad X \to AV \mid BX \qquad \qquad V \to ZB$$
$$A \to a \qquad \qquad B \to b$$

\clearpage


\section*{Exercise \homeworkNumber.2}
\begin{proof}
Length of Derivations in Chomsky Normal Form:\\\\
In the case $\vert w \vert = 1$ we can only have one terminal Rule $ S \to a $ because $ S \to AB $ would be replaced by at least two terminal rules and therefore be $\vert w \vert \geq 2$. \\\\
With $ S \to AB $ we can generate any word $w$ with $\vert w \vert \geq 2$ in the language $G$ by replacing the $A$ and (NOT SURE IF OR WAS CORRECT!) $B$.
$$S \to AB \xrightarrow{(B \to CD)} ACD \xrightarrow{(D \to EF)} ACEF \to \dots \to X_1 \dots X_n$$
By the definition of the Chomsky Normal Form, the Grammar has either a terminal or two Variables on the right-hand side. Therefore with every Step, exactly one Variable can be added.\\
So to generate $X_1 \dots X_n$ exactly $n-1$ steps are needed.\\\\
Now we have to replace every Variable $X$ with a terminal symbol. By definition we can only replace one $X$ at a time and therefore need $n$ more steps to generate our final word $w$.\\\\
In total we need exactly $2n-1$ steps and since $n = \vert w \vert$ the statement is true.\\
\end{proof}


\section*{Exercise \homeworkNumber.3}
\begin{enumerate}[(a)]
	\item
	\renewcommand{\arraystretch}{1.25}
	\begin{tabular}{ c | c | l }
	$\Sigma$ read input & step & Stack status\\
	\hline
	$\epsilon$ & $q_0 \to q_1$ & $\#$\\
	a & $q_1 \to q_1$ & $\#X$\\
	$\epsilon$ & $q_1 \to q_2$ & $\#X$\\
	a & $q_2 \to q_2$ & $\#XY$\\
	a & $q_2 \to q_2$ & $\#XYY$\\
	d & $q_2 \to q_3$ & $\#XYY$\\
	b & $q_3 \to q_4$ & $\#XYY$\\
	a & $q_4 \to q_3$ & $\#XY$\\
	b & $q_3 \to q_4$ & $\#XY$\\
	a & $q_4 \to q_3$ & $\#X$\\
	$\epsilon$ & $q_3 \to q_5$ & $\#X$\\
	c & $q_5 \to q_6$ & $\#X$\\
	c & $q_6 \to q_5$ & $\#$\\
	$\epsilon$ & $q_5 \to q_7$ & $\epsilon$\\
	\end{tabular}
	
	\item
	\end{enumerate}
	This PDA accepts the following language: 
	$w = \set{a^x \; d \; (ba)^y \; (cc)^z \mid a = y+z,\quad x,y,z \in \mathbb{N}_0}$\\
	The self-loops $q_1$ and $q_2$ stack $a$'s. Then follows a single $d$ and the loop $q_3 \to q_4 \to q_3$ ($ab$) which removes anything stacked by $q_1$ followed by the loop $q_5 \to q_6 \to q_5$ ($cc$) which removes anything stacked by $q_2$. So in the end the visits of $q_1 + q_2$ have to equal the visits of $q_3 + q_5$.


\section*{Exercise \homeworkNumber.4}

\end{document}
