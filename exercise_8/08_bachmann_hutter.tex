%% General definitions
\documentclass{article} %% Determines the general format.
\usepackage{a4wide} %% paper size: A4.
\usepackage[utf8]{inputenc} %% This file is written in UTF-8.
%% Some editors on Windows cannot save files in UTF-8.
%% If there is a problem with special characters not showing up
%% correctly, try switching "utf8" to "latin1" (ISO 8859-1).
\usepackage[T1]{fontenc} %% Format of hte resulting PDF file.
\usepackage{fancyhdr} %% Package to create a header on each page.
\usepackage{lastpage} %% Used for "Page X of Y" in the header.
											%% For this to work, you have to call pdflatex twice.
\usepackage{enumerate} %% Used to change the style of enumerations (see below).

\usepackage{amssymb} %% Definitions for math symbols.
\usepackage{amsmath} %% Definitions for math symbols.
\usepackage{amsthm}
\usepackage{braket}
\usepackage{graphicx}
\usepackage{float}
\usepackage{hyperref}
\usepackage{algorithm}
\usepackage{algorithmic}

\usepackage{tikz}  %% Pagacke to create graphics (graphs, automata, etc.)
\usetikzlibrary{automata} %% Tikz library to draw automata
\usetikzlibrary{arrows}   %% Tikz library for nicer arrow heads


%% Left side of header
\lhead{\course\\\semester\\Exercise \homeworkNumber}
%% Right side of header
\rhead{\authorname\\Page \thepage\ of \pageref{LastPage}}
%% Height of header
\usepackage[headheight=36pt]{geometry}
%% Page style that uses the header
\pagestyle{fancy}

\newcommand{\authorname}{Nico Bachmann\\Ruben Hutter}
\newcommand{\semester}{Spring semester 2023}
\newcommand{\course}{Theory of Computer Science}
\newcommand{\homeworkNumber}{8}


\begin{document}

\section*{Exercise \homeworkNumber.1}

\begin{enumerate}[(a)]
\item $S = \set{f \mid f \text{ is undefined for input } 0}$

\item $S = \set{f \mid f \text{ is a total function and computable}}$

\item There exists a TM that accepts the same language as a another TM but one halts on an even number of steps while the other on an uneven number of steps.

\item $S = \lbrace f \mid f$ computes the binary multiplication function mul: $\mathbb{N}_0^2 \to \mathbb{N}_0 \text{ with mul(x, y) = x * y} \rbrace$
\end{enumerate}

\section*{Exercise \homeworkNumber.2}
Since there exist functions to transform a given type-0 grammar to a DTM and vice versa, we can look at the problem from a DTM perspective.\\
Let $S = \lbrace L(M) \vert L(M) = \emptyset \rbrace$ be the set of languages that contains all languages of TM's that have an empty language.\\

Let $M_1$ be a DTM with $L(M_1) = \emptyset$. For example $M_1$ could reject every input immediately. Then let $M_2$ be a DTM with $L(M_2) \neq \emptyset$. For example $M_2$ could accept a single string.\\

In this case, $L(M_1) \in S$ but $L(M_2) \notin S$ and we see that S is a non-trivial property of Turing Machine languages.\\

Now Rice theorem tells us that every non-trivial property of a language is undecidable and because there exist functions to transform from a DTM to a type-0 grammar, we can apply this result also for the type 0 grammars. Therefore $EMPTINESS$ is in fact undecidable.

\section*{Exercise \homeworkNumber.3}
\begin{enumerate}[(a)]
\item
A non-deterministic algorithm for the Hitting Set problem is as follows:

\begin{algorithm}
\caption{Non-deterministic Algorithm for HittingSet (short version)}
\begin{algorithmic}[1]
\STATE Choose a set H of size at most $k$ non-deterministically.
\FOR{each set $S_i \in S$}
	\STATE Check whether $S_i \cap H$ is non-empty.
	\IF{$S_i \cap H$ is empty}
		\STATE Reject the instance.
	\ENDIF
\ENDFOR
\STATE Accept the instance.
\end{algorithmic}
\end{algorithm}

The algorithm runs in polynomial time because each non-deterministic choice can be made in polynomial time. The size of the input is proportional to the size of the set U and the number of sets in S, so the runtime is polynomial in the size of the input.

The algorithm is correct because if there exists a hitting set of size at most k, then the algorithm will accept the instance. If there does not exist a hitting set of size at most k, then the algorithm will reject the instance for any choice of H of size at most k. Therefore, the algorithm correctly decides the decision version of the Hitting Set problem.

\item
\begin{algorithm}
\caption{Deterministic Algorithm for HittingSet (short version) xD}
\begin{algorithmic}[1]
\FOR{each $H' \in$ all possible subsets of $U$ with $|H'| \leq k$}
    \STATE Check if $S_i \cap H' \neq \emptyset$ for all $Si \in S$.
    \RETURN True
\ENDFOR
\RETURN False
\end{algorithmic}
\end{algorithm}

\end{enumerate}

\section*{Exercise \homeworkNumber.4}
\begin{enumerate}[(a)]
\item $n^3$ will dominate as $n$ gets larger. Therefore the runtime of X is bound to a polynomial function and we can conclude that X is a P problem.
\item The runtime of $n^{log_2(n)}$ grows faster than any polynomial function because the exponent is not constant and grows with the size of n. We cannot conclude that X is in P for that reason.
\end{enumerate}


\end{document}
