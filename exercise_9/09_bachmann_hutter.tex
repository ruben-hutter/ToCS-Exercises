%% General definitions
\documentclass{article} %% Determines the general format.
\usepackage{a4wide} %% paper size: A4.
\usepackage[utf8]{inputenc} %% This file is written in UTF-8.
%% Some editors on Windows cannot save files in UTF-8.
%% If there is a problem with special characters not showing up
%% correctly, try switching "utf8" to "latin1" (ISO 8859-1).
\usepackage[T1]{fontenc} %% Format of hte resulting PDF file.
\usepackage{fancyhdr} %% Package to create a header on each page.
\usepackage{lastpage} %% Used for "Page X of Y" in the header.
											%% For this to work, you have to call pdflatex twice.
\usepackage{enumerate} %% Used to change the style of enumerations (see below).

\usepackage{amssymb} %% Definitions for math symbols.
\usepackage{amsmath} %% Definitions for math symbols.
\usepackage{amsthm}
\usepackage{braket}
\usepackage{graphicx}
\usepackage{float}
\usepackage{hyperref}
\usepackage{algorithm}
\usepackage{algorithmic}

\usepackage{tikz}  %% Pagacke to create graphics (graphs, automata, etc.)
\usetikzlibrary{automata} %% Tikz library to draw automata
\usetikzlibrary{arrows}   %% Tikz library for nicer arrow heads


%% Left side of header
\lhead{\course\\\semester\\Exercise \homeworkNumber}
%% Right side of header
\rhead{\authorname\\Page \thepage\ of \pageref{LastPage}}
%% Height of header
\usepackage[headheight=36pt]{geometry}
%% Page style that uses the header
\pagestyle{fancy}

\newcommand{\authorname}{Nico Bachmann\\Ruben Hutter}
\newcommand{\semester}{Spring semester 2023}
\newcommand{\course}{Theory of Computer Science}
\newcommand{\homeworkNumber}{9}


\begin{document}

\section*{Exercise \homeworkNumber.1}


\section*{Exercise \homeworkNumber.2}
\begin{enumerate}[(a)]
\item
No, the statement is not correct. A language can be in both P and NP because P represents problems solvable in polynomial time, while NP represents problems verifiable in polynomial time. Being verifiable in polynomial time does not preclude a problem from also being solvable in polynomial time, thus allowing a language to belong to both classes.

\item
The statement is not necessarily true. If $X$ is an NP-complete problem and $Y$ is a problem with $X \leq Y$ (meaning $Y$ is polynomial-time reducible to $X$), it does not automatically imply that $Y$ is NP-complete. The NP-completeness of a problem is not preserved under polynomial-time reductions in the general case.
\end{enumerate}

\section*{Exercise \homeworkNumber.3}
I cannot prove that all languages in NP are decidable because it is not true. In fact, the question of whether P (class of problems decidable in polynomial time) is equal to NP is an open problem in computer science. If P = NP were proven, it would imply that all languages in NP are decidable. However, as of now, it remains an unsolved question.

\end{document}
