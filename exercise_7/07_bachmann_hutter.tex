%% General definitions
\documentclass{article} %% Determines the general format.
\usepackage{a4wide} %% paper size: A4.
\usepackage[utf8]{inputenc} %% This file is written in UTF-8.
%% Some editors on Windows cannot save files in UTF-8.
%% If there is a problem with special characters not showing up
%% correctly, try switching "utf8" to "latin1" (ISO 8859-1).
\usepackage[T1]{fontenc} %% Format of hte resulting PDF file.
\usepackage{fancyhdr} %% Package to create a header on each page.
\usepackage{lastpage} %% Used for "Page X of Y" in the header.
											%% For this to work, you have to call pdflatex twice.
\usepackage{enumerate} %% Used to change the style of enumerations (see below).

\usepackage{amssymb} %% Definitions for math symbols.
\usepackage{amsmath} %% Definitions for math symbols.
\usepackage{amsthm}
\usepackage{braket}
\usepackage{graphicx}
\usepackage{float}
\usepackage{hyperref}

\usepackage{tikz}  %% Pagacke to create graphics (graphs, automata, etc.)
\usetikzlibrary{automata} %% Tikz library to draw automata
\usetikzlibrary{arrows}   %% Tikz library for nicer arrow heads


%% Left side of header
\lhead{\course\\\semester\\Exercise \homeworkNumber}
%% Right side of header
\rhead{\authorname\\Page \thepage\ of \pageref{LastPage}}
%% Height of header
\usepackage[headheight=36pt]{geometry}
%% Page style that uses the header
\pagestyle{fancy}

\newcommand{\authorname}{Nico Bachmann\\Ruben Hutter}
\newcommand{\semester}{Spring semester 2023}
\newcommand{\course}{Theory of Computer Science}
\newcommand{\homeworkNumber}{7}


\begin{document}

\section*{Exercise \homeworkNumber.1}
\begin{enumerate}[(a)]
\item
$$
x_1 = \texttt{dec}(101) = 5,\qquad x_2 = \texttt{dec}(0) = 0,\qquad x_3 = \texttt{dec}(11) = 3,\qquad y = \texttt{dec}(1001) = 9
$$
\item
$$
x_1 = \texttt{dec}(11) = 3,\qquad x_2 = \texttt{dec}(100) = 4,\qquad x_3 = \texttt{dec}(1) = 1,\qquad y = \textit{undefined}
$$
\end{enumerate}

\section*{Exercise \homeworkNumber.2}
Read hints

\section*{Exercise \homeworkNumber.3}
To show that the composition $(f \circ g) : \Sigma_1^* \to \Sigma_2^*$ is Turing-computable, we can build a TM that simulates the computations of f and g sequentially. This TM would work as follows:
\begin{enumerate}
\item
Start the TM with the input $x$
\item
Use a Turing machine that simulates the computation of $g$ on $x$. If $g(x)$ is undefined, then halt and output "undefined".
\item
Use a Turing machine that simulates the computation of $f$ on $g(x)$. If $f(g(x))$ is undefined, then halt and output "undefined"
\end{enumerate}

If g(x) is undefined -> undefined
\clearpage


\section*{Exercise \homeworkNumber.4}


\section*{Exercise \homeworkNumber.5}


\end{document}
